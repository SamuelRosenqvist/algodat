\documentclass{tufte-handout}
\usepackage{amsmath}
\usepackage[utf8]{inputenc}
\usepackage{mathpazo}
\usepackage{booktabs}
\usepackage{microtype}

\pagestyle{empty}


\title{Flow Report}
\author{Comrade Samuel Rosenqvist}

\begin{document}
  \maketitle

  \section{Results}

  Our implementation successfully computes a flow of 163 on the input file, confirming the analysis of the American enemy.
  

  We have analysed the possibilities of descreasing the capacities near Minsk.
  Our analysis is summarized in the following table:
  

\bigskip
  \begin{tabular}{rccc}\toprule
    Case & 4W--48 & 4W--49 & Effect on flow \\\midrule
    1& 30& 20 & no change \\
    2& 20 &30 & no change \\
    3&20 & 20& no change \\
    4&20 & 10& $-10$ \\
    5&10 & 20& $-10$ \\ 
    6&10 & 10& $-20$ \\ \\\bottomrule
  \end{tabular}
  \bigskip

  In case 4, the new bottleneck becomes
  \begin{quote}
      10--25, 11--24, 11--25, 12-23, 18--22, 20--21, 20--23, 27--26, 28--30
  \end{quote}
  The comrade from Minsk is advised to focus resources on one of the two lines to secure the overall capacity of the railway. 

  \section{Implementation details}

  We use a straighforward implemenation of Ford-Fulkerson method.
  We use a dfs algorithm to find an augmenting path.

  The running time is $O(E*f)$ where $E$ is the number of edges and $f$ is the maximum flow in the graph.

  The edges and the residual graph are represented as adjacency matrices built from nested lists.
  Below is initialization of the residual graph where all edges are given a initial flow of zero.
  
  \begin{verbatim}
    
      F = [[0] * n for i in range(n)]
    
  \end{verbatim}


\end{document}
